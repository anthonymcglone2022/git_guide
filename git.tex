%==========================TOC tips by Chas====================================
\documentclass[12pt,a4paper]{report}

\usepackage[margin=1in]{geometry}
\usepackage{lipsum}
\usepackage{times}

\setcounter{secnumdepth}{3}
\setcounter{tocdepth}{3}
\renewcommand{\contentsname}{Table of Contents}

\usepackage{titletoc}

%\titlecontents{section}[left]{above-code}{numbered-entry-format}{numberless-entry-format}{filler-page-format}[below-code]
\newcommand{\setupname}[1][\chaptername]{
\titlecontents{chapter}[0pt]{\vspace{1ex}}{\bfseries#1~\thecontentslabel:\quad}{\bfseries}{\bfseries\hfill\contentspage}[]
}

\author{Anthony McGlone}\title{git - Command Line Guide}
\begin{document}\maketitle

\tableofcontents

\setupname
\chapter{Introduction}
\section{What is git?}
git is a repository (or store) which is used to store files. git "versions" the files in the repository, so every time a commit (or save) operation is made, the state of the files in that repository is saved (with a unique ID). This means that you can see the history of the changes in the repository. You can also delete the changes in the repository by going back to a specific commit. Git is mostly used by Software Engineers to store and version code for software releases. It's a powerful tool for collaboration on large projects. This guide will demonstrate how the git command line tool is used to manage a repository.  

\section{Installing git}
\subsection{MacOS}

1. Install Homebrew (https://brew.sh/) if you don't have it (Check by opening a terminal and running "brew -h")
\newline
2. Install git by running "brew install git"
\newline
3. Open a terminal and run "git \texttt{--}version" (if git is installed, the version number should be printed to console) 

\subsection{Windows}

1. Install git by downloading the Windows binary from git-scm (https://git-scm.com/) 
\newline
2. Open a MS-DOS terminal and run "git \texttt{--}version" (if git is installed, the version number should be printed to console)


\subsection{Linux (Ubuntu)}

1. Install git by opening a terminal and running "sudo apt-get install git"
\newline
2. Run "git \texttt{--}version" (if git is installed, the version number should be printed to console)


\subsection{Linux (Red Hat)}


1. Install git by opening a terminal and running "sudo yum install git" 
\newline
2. Run "git \texttt{--}version" (if git is installed, the version number should be printed to console)





\end{document}



















