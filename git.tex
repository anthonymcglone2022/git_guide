\documentclass[12pt,a4paper]{report}

\usepackage[margin=1in]{geometry}
\usepackage{lipsum}
\usepackage{times}
\usepackage{listings}
\usepackage{hyperref}


\setcounter{secnumdepth}{3}
\setcounter{tocdepth}{3}
\renewcommand{\contentsname}{Table of Contents}

\usepackage{titletoc}

\newcommand{\setupname}[1][\chaptername]{
\titlecontents{chapter}[0pt]{\vspace{1ex}}{\bfseries#1~\thecontentslabel:\quad}{\bfseries}{\bfseries\hfill\contentspage}[]
}

\author{Anthony McGlone}\title{Git - Command Line Guide}
\begin{document}\maketitle

\tableofcontents

\setupname
\chapter{Introduction}
\section{What is Git?}

Git is a repository which is used to store files. Git is used by Software Engineers to store and version code for software releases. It's a powerful tool for collaborating on large projects.
\newline
\newline
Git versions the files within its repository, so every time a commit operation is made, the files in the repository are saved - a unique ID is also created and associated with the commit. This means that you can see the history of the changes in the repository. You can also delete the changes in the repository by going back to a specific commit.
\newline
\newline
In this guide you will see how the command line tool \texttt{git} is used to manage this repository.  

\section{Installing Git}
\subsection{MacOS}

1. Open a terminal. Then install Homebrew (steps located: \href{https://brew.sh/}{here}).
\newline
2. Install Git by running \texttt{brew install git}
\newline
3. Open a terminal and run \texttt{git --version} (if Git is installed, the version number should be printed to console) 

\subsection{Windows}

1. Install Git by downloading the Windows binary from git-scm (\href{https://git-scm.com/}{here})
\newline
2. Run the installer
\newline
3. Open a Git Bash terminal and run \texttt{git --version} (if Git is installed, the version number should be printed to console)


\subsection{Linux (Ubuntu)}

1. Install Git by opening a terminal and running \texttt{sudo apt-get install git}
\newline
2. Run \texttt{git --version} (if Git is installed, the version number should be printed to console)


\subsection{Linux (Red Hat)}


1. Install git by opening a terminal and running \texttt{sudo yum install git} 
\newline
2. Run \texttt{git --version} (if Git is installed, the version number should be printed to console)


\chapter{Basic Git command line operations}
\section{Setting up a local repository}
First create a folder called \texttt{store}. Then open a terminal and navigate into that folder. Run this command:
\newline
\newline
\centerline{\texttt{git init}}
\newline
\newline
This command creates the repository and also creates a \texttt{.git}  subdirectory. This subdirectory stores information about commits, and the location of your remote repository. This remote repository is stored on a server (hosted on the internet or on a company's internal network). To share your commits with others, you have to push your local commits to the remote repository. We'll set up the remote repository later.
\newline
\newline
Let's add your username and email to your local repository now. This will be required so your commits can be associated with you. Start with the username. Run the following command (before doing so, add your full name between the quotes in the command below):
\newline
\newline
\centerline{\texttt{git config --global user.name "INSERT NAME HERE"}}
\newline
\newline
Now run the command to set up your email (again, insert your email address between the quotes):
\newline
\newline
\centerline{\texttt{git config --global user.email "INSERT EMAIL HERE"}}
\newline
\newline
Now run the command \texttt{git config --global user.name}. If everything is correct, you should see your username. Run \texttt{git config --global user.email} to see if your email prints out to console.
\newline
\newline
Run the following command to verify that your local repository was set up:
\newline
\newline
\centerline{\texttt{git status}}
\newline
\newline
You should see the following text in your terminal:
\newline
\newline
\newline
\texttt{
On branch master
\\
\\
No commits yet
\\
\\
nothing to commit (create/copy files and use "git add" to track)
}
\newline
That's it! Your local repository is now successfully set up. The branch \texttt{master} is the main branch in your local repository. By default, it's the place where your files are stored. You can create other branches (basically copies of \texttt{master}) and work on edits there before saving them back into the \texttt{master} branch. For now, we'll just work with the \texttt{master} branch. We'll cover branching strategies and remote branches (stored in a remote repository) later.


\section{Committing files into the \texttt{store} repository}

First, create a \texttt{code.txt} file in the \texttt{store} folder. Then run \texttt{git status} in your terminal. You should see the following output:
\newline
\newline
\texttt{
Untracked files:
\\
\hspace*{10mm}(use "git add <file>..." to include in what will be committed)
\\
\\
\hspace*{20mm}code.txt
}
\newline
\newline
The file is untracked, meaning Git doesn't see it yet. If you tried to do a commit operation now, Git wouldn't save the file or version it. Run the add operation below in your terminal to get Git to track the file. 
\newline
\newline
\centerline{\texttt{git add code.txt}}
\newline
\newline
Now run \texttt{git status} again to see the tracking:
\newline
\newline
\texttt{
On branch master
\\
\\
No commits yet
\\
\\
Changes to be committed:
\\
\hspace*{10mm}(use "git rm --cached <file>..." to unstage)
\\
\\
\hspace*{20mm}new file:   code.txt
}
\newline
\newline
Now commit these changes by running \texttt{git commit -m "This is my first commit!"} You can run \texttt{git log} after to see the commit ID. It should look something similar to this:
\newline
\newline
\texttt{
commit b973b70df3ffcfd8dbe0284c76e0d3bd7c30f3b6 (HEAD -> master)
\\
Author: anthonymcglone2022 <anthonymcglone2022@gmail.com>
\\
Date:   Wed May 11 20:14:36 2022 +0100
\\
\\
\hspace*{10mm}This is my first commit
}
\newline
\newline
\section{Creating a remote repository}
GitHub is a website that offers free hosting of remote Git repositories. In order to create a repository there, there are a few steps that have to be completed. 
\newline

- Signup to GitHub: \href{https://github.com/signup}{here} 

- Generate an SSH key: \href{https://docs.github.com/en/authentication/connecting-to-github-with-ssh/generating-a-new-ssh-key-and-adding-it-to-the-ssh-agent}{Procedure here}

- Adding that SSH key into GitHub: \href{https://docs.github.com/en/authentication/connecting-to-github-with-ssh/adding-a-new-ssh-key-to-your-github-account}{Procedure here}

- Create a repository (name it \texttt{store} and skip the README initilization step.
\newline
\hspace*{9mm}Stop at the create instruction): \href{https://docs.github.com/en/get-started/quickstart/create-a-repo}{Procedure here}
\newline
\newline
Let's push the commit we made to the remote repository. First, we will point our local repo to the remote repo. Take your GitHub user name and replace \texttt{username} in the following command - then run it.
\newline
\newline
\centerline{\texttt{git remote add origin git@github.com:username/store.git}}
\newline
\newline
To see if the repository was updated, run \texttt{git remote -v}








\end{document}



















